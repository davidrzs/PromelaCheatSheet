\documentclass{sciposter}
\usepackage[cmyk,dvipsnames,usenames,svgnames,table]{xcolor} 
\usepackage{lipsum}
\usepackage{epsfig}
\usepackage{amsmath}
\usepackage{amssymb}
\usepackage[english]{babel}
\usepackage{geometry}
\usepackage{multicol}
\usepackage{graphicx}
\usepackage{tikz}
\usepackage{wrapfig}
\usepackage{gensymb}
\usepackage[utf8]{inputenc}
\usepackage{empheq}
\usepackage[pdfusetitle]{hyperref}
\usepackage{fancyvrb}
\usepackage{xtab}

\usepackage{graphicx,url}
\usepackage{palatino}

\geometry{
 landscape,
 a1paper,
 left=5mm,
 right=50mm,
 top=5mm,
 bottom=50mm,
 }




\providecommand{\tightlist}{%
	\setlength{\itemsep}{0pt}\setlength{\parskip}{0pt}}

%BEGIN LISTINGDEF

\usepackage{listings}
   
\usepackage{inconsolata}

\definecolor{background}{HTML}{fcfeff}
\definecolor{comment}{HTML}{b2979b}
\definecolor{keywords}{HTML}{255957}
\definecolor{basicStyle}{HTML}{6C7680}
\definecolor{variable}{HTML}{001080}
\definecolor{string}{HTML}{c18100}
\definecolor{numbers}{HTML}{3f334d}

\lstdefinestyle{lectureNotesListing}{
	xleftmargin=0.5em, % 2.8 with line numbers
	breaklines=true,
	frame=single,
	framesep=0.6mm,
	frameround=ffff,
	framexleftmargin=0.4em, % 2.4 with line numbers | 0.4 without them
	tabsize=4, %width of tabs
	aboveskip=0.5em,
	classoffset=0,
	sensitive=true,
	% Colors
	backgroundcolor=\color{background},
	basicstyle=\color{basicStyle}\small\ttfamily,
	keywordstyle=\color{keywords},
	commentstyle=\color{comment},
	stringstyle=\color{string},
	numberstyle=\color{numbers},
	identifierstyle=\color{variable},
	showstringspaces=true
}
\lstset{style=lectureNotesListing}

%END LISTINGDEF


%BEGIN TODO DEFINITION
\usepackage{todonotes}
\newcommand{\TODO}[1]{\todo[inline]{\Large TODO:  #1}}
\newcommand*\widefbox[1]{\fbox{\hspace{2em}#1\hspace{2em}}}
%END TODO DEFINITION


%BEGIN OTHER COMMANDS
\renewcommand{\t}[1]{\texttt{#1}}

%BEGIN SETUP
\title{\huge{Promela}}


\author{\Large Cheatsheet}
%END SETUP

% set font
\renewcommand{\familydefault}{\rmdefault}


\begin{document}


\maketitle

\begin{multicols}{3}

\section*{Functions and Initialization}
Promela provides three different abstractions for functions:

\begin{itemize}
	\item \texttt{init}: Used to initialize a Promela model:
	\begin{lstlisting}[language=promela]
init {
	printf("Hello World\n")
}
\end{lstlisting}


	\item \texttt{proctype}: A simple process that can take parameters (except arrays)
\begin{lstlisting}[language=promela]
proctype myProc(int p){...}
\end{lstlisting}
the process above must be run from e.g. an \texttt{init} function with \texttt{run myProc(42);}. The \texttt{run} command start the procedure and returns immediately.

Alternatively we can also start procedures right from the initial system state using the active keyword:
\begin{lstlisting}[language=promela]
active proctype myProc(int p){...}
\end{lstlisting}
or start \texttt{N} processes of the type \texttt{myProc}.
\begin{lstlisting}[language=promela]
active [N] proctype myProc(int p){...}
\end{lstlisting}
{\color{red}If we pass multiple variables we seperate them with a semicolon and not a comma!}
\item \texttt{inline}: A macro that gets copy-pasted. Doesn't do any simulation but just replaces text for a symbolic name with possible parameters:
\begin{lstlisting}[language=promela]
inline swap(a, b) {
	int tmp;
	tmp = a;
	a = b;
	b = tmp;
}
\end{lstlisting}
Note that we do not have a return value and cannot use recursion as well as we do not have a new variable scope as we are literally copy-pasting.
\end{itemize}

\section*{Synchronization \& Assertions}
\subsection*{Atomic Block}
To bundle multiple instructions together and have them execute atomically, we can bundle them as follows:

\begin{lstlisting}[language=promela]
atomic{
	printf("hello ");
	printf("world\n");
}
\end{lstlisting}

\subsection*{Guards}
We can suspend the execution until a condition is true. This is called a guard and is a boolean expression as a statement. The statement thereafter will only be executed if the guard evaluates to true:

\begin{lstlisting}[language=promela]
time == 42;
printf("it is time\n");
\end{lstlisting}

\subsection*{Assertion}
If we want to model check a program we can add assertions. The model checker will then check all combinations for a violation of the assertions:

\begin{lstlisting}[language=promela]
proctype finish(){
	finished == 0; /* wait until finished */
	assert(n == 100);
}
\end{lstlisting}


\section*{Control Flow}

\subsection*{If-Statement}
One of the guard that is true is chosen non-deterministically. One (and only one) option can be \texttt{else} which is taken if no other match. If no conditions matched spin will block until a matching one could be found.
\begin{lstlisting}[language=promela]
if
:: x > 0 -> x--
:: x < 0 -> x++
:: x > 100 -> x += 100
:: else  -> break
fi
\end{lstlisting}




\subsection*{Do-Statement}
This loop construct executes an option non-deterministically until a \texttt{break} or \texttt{goto} is found.
\begin{lstlisting}[language=promela]
do
:: count++
:: count--
:: (count == 0) -> break
od
\end{lstlisting}

\section*{Declarations}
\subsection*{Constants}
Define a constant to be replaced: The two are completely equivalent:
\begin{lstlisting}[language=promela]
mtype = { ack, nak, err, next, accept };
\end{lstlisting}
\begin{lstlisting}[language=promela]
#define	ack	5
#define nak	4
#define	err	3
#define next	2
#define accept	1
\end{lstlisting}

\subsection*{Structures}
\begin{lstlisting}[language=promela]
typedef vector {int x; int y};
\end{lstlisting}

\subsection*{Variable}
Can be global or scoped:
\begin{lstlisting}[language=promela]
bit counter;
byte counter2 = 2;
\end{lstlisting}
We can choose between \texttt{bit, bool, byte, short, int} as well as arrays which we create like in C: \texttt{int array[2]}
\subsection*{Built in Primitives}
\texttt{len(), empty(), nempty(), nfull(), full(), run eval() enabled() pcvalue()} are built in functions. Furthermore we have the standard C operators. 

A special addition is the \texttt{\_pid} variable giving me the process id of a single process an the \texttt{\_nr\_pr} giving me the total number of processes.


\section*{Channels}

We can create a channel using
\begin{lstlisting}[language=promela]
chan ch = [d] of {mtype, int}
\end{lstlisting}
That channel takes two parameters, an mtype and an int. The \texttt{d} is the buffer size. If it is unbuffered (d=0) we can use the channel for synchronization as both threads have to rendevouz to exchange a message. Otherwise we will just place it in the buffer.
\begin{itemize}
	\item Receiving: \texttt{ch ? req, n;} receives a \texttt{req} mtype of arbitrary value \texttt{n}.
	\item Sending: \texttt{ch ! ack, 1;} send an \texttt{ack} mtype with value \texttt{1}.
\end{itemize} 


\section*{Sources}

\begin{enumerate}
	\item \href{https://infsec.ethz.ch/education/ss2020/fmfp.html}{ETH lecture slides by Prof. Peter Müller}
	\item \href{https://spinroot.com/spin/Man/mtype.html}{SPIN documentation}
	\item \href{http://www.cse.chalmers.se/edu/year/2017/course/TDA384_LP1/files/exercises/promela-tutoria1l-shared-mem-conc.pdf}{A tutorial introduction to shared memory concurrency by K.V.S. Prasad (Chalmers)}
\end{enumerate}




\newpage


\end{multicols}
\end{document}